\cvsection{Research Projects}

\vspace{-8mm}

\begin{cventries}

    \cventry
      {Research Intern | \href{https://dex-robot.com}{DexRobot Inc.}} % Title
      {Task-Centric Reinforcement Learning for High-DOF Dexterous Hands} % Project name
      {Shanghai} % Location
      {Feb. 2025 -- Present} % Update as needed
      {
        Advancing dexterous manipulation through task-centric RL applied to high-DOF robotic hands. Leveraging real-to-sim human demonstrations and large-scale datasets for robust skill transfer.
        \begin{cvitems}
          \item{\textbf{RL-based dexterous retargeting algorithm} (proposed; first author): Pioneered a novel retargeting method to map human demonstrations to robot morphology while preserving task semantics, achieving 30\% increase in real-world teleoperation efficiency compared to DexPilot baseline}
          \item{\textbf{Large-scale real+synthetic manipulation dataset}: Contributed to RL methodology development for physical plausibility recovery of human demonstrations; building dataset of 1M+ cross-embodiment demonstrations for classified grasping tasks on dexterous hands of various morphology}
          \item{\textbf{Simulation setup}: Constructed simulation scenes in MuJoCo and Isaac Gym; deployed benchmark datasets such as ARCTIC to evaluate manipulation accuracy and generalization}
          \item{\textbf{Policy deployment}: Implemented hardware and control integration for the 19-DOF DexHand 021, enabling adaptive in-hand manipulation under dynamic contact conditions}
        \end{cvitems}
      }
  \cventry
    {Engineering Intern | \href{https://dex-robot.com}{DexRobot Inc.}} % Title
    {Upper Limb 60-DOF Data Acquisition and Digital Twin System} % Project name
    {Shanghai} % Location
    {Sep. 2024 -- Mar. 2025} % Date
    {
      Engineered a 60-DOF upper-limb capture system that enables real-time teleoperation for high-fidelity human demonstration capture
      \begin{cvitems}
        \item{\textbf{Mechanical Design:} Designed lightweight exoskeleton using 3D-printed nylon structures, integrating magnetic encoders for joint angle sensing and smart gloves with tactile feedback for hand motion capture}
        \item{\textbf{Embedded Systems:} Implemented Socket and CANFD bus communication for low-latency peripheral device control and data streaming, achieving 10ms end-to-end latency}
        \item{\textbf{Robot Kinematics \& Control:} Implemented kinematic modeling and mapping algorithms to reconstruct operator motion. Created Unity3D-based digital twin environment for real-time teleoperation of dual-arm JAKA robots and DexHand 021 systems}
      \end{cvitems}
    }
    \cventry
      {National First Prize, Mechanical Innovation Design Competition | Advisor: Gang Wang} % Title
      {Bionic Robotic Peacock} % Project name
      {Changsha} % Location
      {May 2022 -- Aug. 2022} % Date
      {
        Engineered a bionic robotic peacock with embedded control, multi-axis actuation, and voice interaction for educational exhibition
        \begin{cvitems}
          \item{Architected embedded control system based on dual STM32 microcontrollers, programmed in embedded C using Keil IDE, with modular architecture supporting coordinated actuation of 11 motors}
          \item{Programmed complex motion behaviors (tail spreading, wing retracting, dancing, walking) through heterogeneous motor control: servo motors for tail, brushless DC motors for neck/wings, worm gear motors for legs}
          \item{Integrated real-time voice control via LU-ASR01 speech recognition module; conducted three design iterations to optimize motion smoothness and structural reliability}
        \end{cvitems}
      \vspace{4mm}
      }
% \vspace{-7mm}
\end{cventries}
